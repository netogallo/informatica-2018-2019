%%%%%%%%%%%%%%%%%%%%%%%%%%%%%%%%%%%%%%%%%
% Programming/Coding Assignment
% LaTeX Template
%
% This template has been downloaded from:
% http://www.latextemplates.com
%
% Original author:
% Ted Pavlic (http://www.tedpavlic.com)
%
% Note:
% The \lipsum[#] commands throughout this template generate dummy text
% to fill the template out. These commands should all be removed when 
% writing assignment content.
%
% This template uses a Perl script as an example snippet of code, most other
% languages are also usable. Configure them in the "CODE INCLUSION 
% CONFIGURATION" section.
%
%%%%%%%%%%%%%%%%%%%%%%%%%%%%%%%%%%%%%%%%%

%----------------------------------------------------------------------------------------
%	PACKAGES AND OTHER DOCUMENT CONFIGURATIONS
%----------------------------------------------------------------------------------------

\documentclass{article}

\usepackage{fancyhdr} % Required for custom headers
\usepackage{lastpage} % Required to determine the last page for the footer
\usepackage{extramarks} % Required for headers and footers
\usepackage[usenames,dvipsnames]{color} % Required for custom colors
\usepackage{graphicx} % Required to insert images
\usepackage{listings} % Required for insertion of code
\usepackage{courier} % Required for the courier font
\usepackage{multirow}
\usepackage{hyperref}

% Margins
\topmargin=-0.45in
\evensidemargin=0in
\oddsidemargin=0in
\textwidth=6.5in
\textheight=9.0in
\headsep=0.25in

\linespread{1.1} % Line spacing

%----------------------------------------------------------------------------------------
%	CODE INCLUSION CONFIGURATION
%----------------------------------------------------------------------------------------

\definecolor{MyDarkGreen}{rgb}{0.0,0.4,0.0} % This is the color used for comments
\lstloadlanguages{c} % Load Perl syntax for listings, for a list of other languages supported see: ftp://ftp.tex.ac.uk/tex-archive/macros/latex/contrib/listings/listings.pdf
\lstset{language=[sharp]c, % Use Perl in this example
        frame=single, % Single frame around code
        basicstyle=\small\ttfamily, % Use small true type font
        keywordstyle=[1]\color{Blue}\bf, % Perl functions bold and blue
        keywordstyle=[2]\color{Purple}, % Perl function arguments purple
        keywordstyle=[3]\color{Blue}\underbar, % Custom functions underlined and blue
        identifierstyle=, % Nothing special about identifiers                                         
        commentstyle=\usefont{T1}{pcr}{m}{sl}\color{MyDarkGreen}\small, % Comments small dark green courier font
        stringstyle=\color{Purple}, % Strings are purple
        showstringspaces=false, % Don't put marks in string spaces
        tabsize=5, % 5 spaces per tab
        %
        % Put standard Perl functions not included in the default language here
        morekeywords={rand},
        %
        % Put Perl function parameters here
        morekeywords=[2]{on, off, interp},
        %
        % Put user defined functions here
        morekeywords=[3]{test},
       	%
        morecomment=[l][\color{Blue}]{...}, % Line continuation (...) like blue comment
        numbers=left, % Line numbers on left
        firstnumber=1, % Line numbers start with line 1
        numberstyle=\tiny\color{Blue}, % Line numbers are blue and small
        stepnumber=5 % Line numbers go in steps of 5
}

\newcommand{\horrule}[1]{\rule{\linewidth}{#1}}

\newcommand\doubleplus{\ensuremath{\mathbin{+\mkern-10mu+}}}

% Creates a new command to include a perl script, the first parameter is the filename of the script (without .pl), the second parameter is the caption
\newcommand{\perlscript}[2]{
\begin{itemize}
\item[]\lstinputlisting[caption=#2,label=#1]{#1}
\end{itemize}
}

\begin{document}

\begin{tabular}{l l}
\multirow{5}{*}{\includegraphics[width=2cm]{../../recursos/logo.png}} & Universidad del Istmo de Guatemala \\
 & Facultad de Ingenieria \\
 & Ing. en Sistemas \\
 & Informatica II \\
 & Prof. Ernesto Rodriguez - \href{mailto:erodriguez@unis.edu.gt}{erodriguez@unis.edu.gt} \\
\end{tabular}
\\\\\\

\begin{center}
        \horrule{0.5pt}
        \huge{Laboratorio \#11} \\
        \large{Fecha de entrega: 25 de Abril, 2019 - 11:59pm} \\
        \horrule{1pt}
\end{center}

\emph{Instrucciones: Resolver cada uno de los ejercicios siguiendo sus respectivas
instrucciones. El trabajo debe ser entregado a traves de Github, en su repositorio del curso, colocado en una carpeta llamada "Laboratorio \#11".
Al menos que la pregunta indique diferente, todas las respuestas a preguntas escritas deben presentarse en
un documento formato pdf, el cual haya sido generado mediante Latex. Este laboratorio
debe ser elaborado en parejas.}

\section*{Tarea \#1 (50\%)}

La funci\'on \emph{fold} es una funci\'on universal que permite replicar
cualquier transformaci\'on de una collecion de elementos de forma general.
Su tipo es el siguiente:
\[
        \forall T\forall S\ .\ S\ fold(\mathtt{std::functcion}\langle S(S,T)\rangle,\ S,\ \mathtt{std::vector}\langle T\rangle) 
\]

Essencialmente, esta funcion toma como parametros una collecion, de valores, un valor inicial
y una funci\'on que le permite combinar un valor del mismo tipo que el valor inicial y un
valor de la coleccion y producir un valor del mismo tipo que el valor inicial. Este processo
se repeite hasta haber consumido todos los valores de la collecion y se haya creado el producto
final. Un ejemplo de su utilizaci\'on es el siguiente:

\perlscript{fold.cc}{}

\section*{Tarea \#2 (50\%)}

La funci\'on \emph{existe} es una funci\'on que permite buscar si un objeto existe dentro de una
colleci\'on. Su tipo es el siguiente:
\[
        \forall T\ .\ \mathtt{bool}\ existe(\mathtt{std::function}\langle \mathtt{bool}(T)\rangle,\ \mathtt{std::vector}\langle T \rangle)
\]
Esta funci\'on acepta un validador y una coleccion de objetos y aplica el validador en cada uno
de los objetos. Si alguno de esos objetos pasa la validaci\'on, la funci\'on retorna \texttt{true},
de lo contrario retorna \texttt{false}. A continuaci\'on se muestra un ejemplo de su utilizaci\'on:
\perlscript{find.cc}{}

Su tarea es utilizar la func\'on \emph{fold} para implementar la funci\'on \emph{existe}.

\section*{Extra \#1 (+2 puntos neto)}

En C++ es possible utilizar cualquier objeto que tenga un metodo llamado \emph{operator()} como
si fuese funci\'on. Por ejemplo:
\perlscript{operator.cc}{}

Sin embargo, la funci\'on \emph{fold}, tal como fue implementada en la tarea \#1 tiene la limitante
que solo puede ser utilizada con instancias de \texttt{std::function}. En otras palabras, el siguiente
codigo no compila:
\perlscript{mal.cc}{}

Debido a que \emph{fold} ha sido llamado con una instancia de \emph{Sumar} en vez de una instancia
de \texttt{std::funciton}. Su tarea es crear una version de \emph{fold} que pueda funcionar con
objetos de {\bf cualquier} tipo siempre y cuando se puedan operar mediante el \emph{operator()}. 

\section*{Extra \#2 (+2 puntos neto)}

Cuando se trabajan proyectos grandes en C++, se acostumbra a automatizar el proceso de compilaci\'on
debido a que el numero de archivos involucrados suele cambiar constantemente. Una herramienta
para facilitar el manejo de proyectos grandes es \href{https://cmake.org}{cmake}. Su tarea es
configurar un archivo CMake para compilar un programa de C++ que utilize el codigo de esta
hoja de trabajo. Como entrega, debe incluir el archivo ``CMakeLists.txt'' en su entrega con
las instrucciones para compilar un ejecutable. Se puede apoyar de \url{https://cmake.org/examples/}
y \url{https://cmake.org/cmake/help/v3.14/manual/cmake-buildsystem.7.html#introduction} para
informaci\'on sobre como utilizar cmake.

\end{document}